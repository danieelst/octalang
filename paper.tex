\documentclass{article}
\usepackage[T1]{fontenc}
\usepackage{fontspec}
\usepackage[a4paper,margin=3cm,top=1cm]{geometry}
\usepackage{tipa}
\usepackage[parfill]{parskip}
\usepackage{float}
\usepackage{amssymb}

\pagenumbering{gobble}
\renewcommand{\thefootnote}{\fnsymbol{footnote}}

\newcommand{\octopus}[0]{{\fontspec{Symbola}\symbol{"1F419}}}
\newcommand{\octalang}[0]{\textsc{Octalang}}

\newcommand{\ipa}[1]{$\langle$\textipa{#1}$\rangle$}
\newcommand{\letter}[1]{{\setmainfont{Noto Serif}#1}}

% Letters
\newcommand{\zero}[0]{\letter{ɵ}}
\newcommand{\one}[0]{\letter{ọ}}
\newcommand{\two}[0]{\letter{o}}
\newcommand{\three}[0]{\letter{\^{o}}}
\newcommand{\four}[0]{\letter{\^{ǫ}}}
\newcommand{\five}[0]{\letter{\^{ọ}}}
\newcommand{\six}[0]{\letter{\^{ɵ}}}
\newcommand{\seven}[0]{\letter{\^{ø}}}

% Brackets
\newcommand{\lmarker}[0]{\letter{(}}
\newcommand{\rmarker}[0]{\letter{)}}

% Divider
\newcommand{\divider}[0]{\letter{/}}

% Compute letter from digit
\newcommand{\digittoletter}[1]{%
  \ifcase#1 \zero{}% Case 0
  \or \one{}% Case 1
  \or \two{}% Case 2
  \or \three{}% Case 3
  \or \four{}% Case 4
  \or \five{}% Case 5
  \or \six{}% Case 6
  \or \seven{}% Case 7
  \else \letter{?}% Default case
  \fi
}

% Compute group from three digit octal value
\newcommand{\octaltogroup}[3]{\lmarker{}\digittoletter{#1}\digittoletter{#2}\digittoletter{#3}\rmarker{}}

% Terms
\newcommand{\tvalue}[0]{\textsc{value}}
\newcommand{\tletter}[0]{\textsc{letter}}
\newcommand{\tdialect}[0]{\textsc{dialect}}
\newcommand{\tgroup}[0]{\textsc{group}}
\newcommand{\tlexeme}[0]{\textsc{lexeme}}
\newcommand{\tsplit}[0]{\textsc{split}}

\begin{document}

\hfill

\begin{center}
  \Huge \octalang{}

  \large \octopus{} Specification of a constructed language \octopus{}

  \large Daniel Stribrand
\end{center}

\hfill

\octalang{} is a constructed language which adopts every other language. It does this by converting a source text's characters' \tvalue{}s\footnote{In accordance to Unicode.} into octal numerals, where each digit corresponds to a \tletter{} in \octalang{}. Thus, \octalang{} inherits the source language's grammar and lexicon. Therefore, every language is considered as a \tdialect{} of \octalang{}. 

\section*{Language}

The alphabet in \octalang{} is an association between the digits in the octal numeral system and the \tletter{}s of \octalang{}.

\begin{table}[H]
  \centering
  \large
  \begin{tabular}{ccc}
    \textbf{Digit} & \textbf{Letter} & \textbf{IPA} \\
    \hline
    0 & \zero{}  & \ipa{u} \\
    1 & \one{}   & \ipa{o} \\
    2 & \two{}   & \ipa{\textlowering{o}} \\
    3 & \three{} & \ipa{O} \\
    4 & \four{}  & \ipa{6} \\
    5 & \five{}  & \ipa{A} \\
    6 & \six{}   & \ipa{\"a} \\
    7 & \seven{} & \ipa{a}
  \end{tabular}
  \caption{\octalang{}'s alphabet}
\end{table}

\tletter{}s are organized into \tgroup{}s by denoting enclosing brackets, semantically a \tgroup{} corresponds to an octal numeral (and thus a character). A succession of non-separated \tgroup{}s is called a \tlexeme{}.

\tlexeme{}s are separated with single \tsplit{}s (\letter{/}).

\section*{Translating to and from {\sffamily\octalang{}}}

The translation to \octalang{} takes the following steps:

\begin{enumerate}
  \item All non-letters from the source text are replaced with \tsplit{}s.
  \item Each symbol is converted into its \tvalue{} as an octal numeral.
  \item The numerals are translated into \octalang{} \tletter{}s and are \tgroup{}ed accordingly.
  \item Preceding and succeeding \tsplit{}s are removed, multiple \tsplit{}s in succession are removed.
\end{enumerate} 

The translation from \octalang{} will naturally be the opposite procedure, though it will result in loss of information due to the homogeneous treatment of non-letters. \tsplit{}s are commonly translated to spaces.

\section*{FAQ\protect\footnote{No one has asked me anything.}}

\begin{itemize}
  \item Is this really a language? Probably not, so do not think too much about it.
  \item How is this language useful? No idea.
\end{itemize}

\newpage
\newgeometry{margin=3cm}

\section*{The English {\sffamily\octalang{}} alphabet}

As an example, we will show the English alphabet as represented in \octalang{}:

\begin{table}[H]
  \centering
  \large
  \begin{tabular}{ccc|ccc}
    \textbf{Upper} & \textbf{Octal value} & \textbf{{\sffamily\textsc{Group}}} & \textbf{Lower} & \textbf{Octal value} & \textbf{{\sffamily\textsc{Group}}} \\
    \hline
    A & 101 & \octaltogroup{1}{0}{1} & a & 141 & \octaltogroup{1}{4}{1} \\
    B & 102 & \octaltogroup{1}{0}{2} & b & 142 & \octaltogroup{1}{4}{2} \\
    C & 103 & \octaltogroup{1}{0}{3} & c & 143 & \octaltogroup{1}{4}{3} \\
    D & 104 & \octaltogroup{1}{0}{4} & d & 144 & \octaltogroup{1}{4}{4} \\
    E & 105 & \octaltogroup{1}{0}{5} & e & 145 & \octaltogroup{1}{4}{5} \\
    F & 106 & \octaltogroup{1}{0}{6} & f & 146 & \octaltogroup{1}{4}{6} \\
    G & 107 & \octaltogroup{1}{0}{7} & g & 147 & \octaltogroup{1}{4}{7} \\
    H & 110 & \octaltogroup{1}{1}{0} & h & 150 & \octaltogroup{1}{5}{0} \\
    I & 111 & \octaltogroup{1}{1}{1} & i & 151 & \octaltogroup{1}{5}{1} \\
    J & 112 & \octaltogroup{1}{1}{2} & j & 152 & \octaltogroup{1}{5}{2} \\
    K & 113 & \octaltogroup{1}{1}{3} & k & 153 & \octaltogroup{1}{5}{3} \\
    L & 114 & \octaltogroup{1}{1}{4} & l & 154 & \octaltogroup{1}{5}{4} \\
    M & 115 & \octaltogroup{1}{1}{5} & m & 155 & \octaltogroup{1}{5}{5} \\
    N & 116 & \octaltogroup{1}{1}{6} & n & 156 & \octaltogroup{1}{5}{6} \\
    O & 117 & \octaltogroup{1}{1}{7} & o & 157 & \octaltogroup{1}{5}{7} \\
    P & 120 & \octaltogroup{1}{2}{0} & p & 160 & \octaltogroup{1}{6}{0} \\
    Q & 121 & \octaltogroup{1}{2}{1} & q & 161 & \octaltogroup{1}{6}{1} \\
    R & 122 & \octaltogroup{1}{2}{2} & r & 162 & \octaltogroup{1}{6}{2} \\
    S & 123 & \octaltogroup{1}{2}{3} & s & 163 & \octaltogroup{1}{6}{3} \\
    T & 124 & \octaltogroup{1}{2}{4} & t & 164 & \octaltogroup{1}{6}{4} \\
    U & 125 & \octaltogroup{1}{2}{5} & u & 165 & \octaltogroup{1}{6}{5} \\
    V & 126 & \octaltogroup{1}{2}{6} & v & 166 & \octaltogroup{1}{6}{6} \\
    W & 127 & \octaltogroup{1}{2}{7} & w & 167 & \octaltogroup{1}{6}{7} \\
    X & 130 & \octaltogroup{1}{3}{0} & x & 170 & \octaltogroup{1}{7}{0} \\
    Y & 131 & \octaltogroup{1}{3}{1} & y & 171 & \octaltogroup{1}{7}{1} \\
    Z & 132 & \octaltogroup{1}{3}{2} & z & 172 & \octaltogroup{1}{7}{2}
  \end{tabular}
  \caption{The English alphabet in \octalang{}.}
\end{table}

\subsection*{{\sffamily\textsc{Hello World}}, as is tradition}

Naturally, we will round off this specification with a traditional \textbf{\sffamily\textsc{Hello World}} example:

\hfill

\begin{center}
  \Large
  % Hello World
  \octaltogroup{1}{1}{0}\octaltogroup{1}{4}{5}\octaltogroup{1}{5}{4}\octaltogroup{1}{5}{4}\octaltogroup{1}{5}{7}\divider{}\octaltogroup{1}{2}{7}\octaltogroup{1}{5}{7}\octaltogroup{1}{6}{2}\octaltogroup{1}{5}{4}\octaltogroup{1}{4}{4}
\end{center}

\vfill

\begin{center}
  $\square$
\end{center}

\end{document}
